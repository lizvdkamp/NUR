\section{Vandermonde matrix}

In this section we look at question 2. 
I have mostly copy-pasted my functions from the tutorials, where I worked together with my sister,
Evelyn van der Kamp (s2138085), so our functions are quite similar.
Also for the iterative function I copied parts from my tutorial functions. 
We are used to coding together and so we have similar techniques.

The script for the entire question is given by:
\lstinputlisting{NURHW1LizQ2.py}

\subsection{Question 2a}

I have created a set of functions to do matrix operations, swapping rows, scaling rows, 
and adding rows to other rows. I also made one function to swap vector rows.
The function to do LU decomposition makes use of an improved version of Crouts algorithm,
including swapping rows to make sure that the biggest number below the diagonal gets put on the diagonal.
The function returns the LU matrix, the solution, and the index array which has information about swapped rows.

I created the Vandermonde matrix and used the LU decomposition function to solve for $c$,
and got the following output:

\lstinputlisting{LU1output.txt}

Afterwards I used the solution to interpolate and get the full 19th-degree polynomial evaluated at
$\approx$ 1000 equally spaced points.
I also used the solution to plot the absolute difference between the data points and the polynomial.

The plot generated can bee seen in Figure \ref{fig:LU1}.

\begin{figure}[h!]
  \centering
  \includegraphics[width=0.9\linewidth]{LU1plot.png}
  \caption{Plot comparing the data points to the 19-degree polynomial obtained with LU
  decomposition. The upper plot shows both, while the lower plot shows the absolute difference
  between the data and the polynomial. This is between 0 and 0.25.}
  \label{fig:LU1}
\end{figure}

\subsection{Question 2b}

In this part I have used my function that uses Neville's algorithm to calculate the Lagrange polynomial
that goes through all of the data points.
It returns the interpolated points that forms the Lagrange polynomial.
I used this function twice to calculate both the 1000 equally spaced points that form the polynomial
as well as the points to calculate the absolute difference between the data and the curve.

The plot generated can bee seen in Figure \ref{fig:Neville}.

\begin{figure}[h!]
  \centering
  \includegraphics[width=0.9\linewidth]{Nevilleplot.png}
  \caption{Plot comparing the data points to the Lagrange polynomial obtained with 
  Neville's algorithm as well as the previously obtained polynomial. The upper plot shows the data and two curves, 
  while the lower plot shows the absolute difference
  between the data and the Lagrange polynomial. This is of order $10^{-14}$.}
  \label{fig:Neville}
\end{figure}

There is a difference between the absolute differences, from 2(a) it was order 0.1 and now it is 
order $10^{-14}$, the polynomials are different, especially at the start.
In the solution from LU decomposition the error is higher because in Neville's algorithm 
we already reduce the error by iterating the different solutions, and we get almost the exact
Lagrange polynomial.

\subsection{Question 2c}

Here I have created a function to iterate over the solution obtained in 2(a) from the LU decomposition.
It calculates the error on the solution by taking $\delta b = A*x - b$ and then solving for 
$A*\delta x = \delta b$ as many times as I give the function.
For this question we iterate over the solution 10 times to obtain a new and improved solution.

The plot generated can bee seen in Figure \ref{fig:LU10}.

\begin{figure}[h!]
  \centering
  \includegraphics[width=0.9\linewidth]{LU10plot.png}
  \caption{Plot comparing the data points to the two curves obtained with LU decomposition.
  LU1 is the initial solution, LU10 shows the solution after iterating 10 times.
  The upper plot shows the data and two curves, 
  while the lower plot shows the absolute difference
  between the data and the ten iterations LU solution. 
  This around 0 for the first couple of points and of order 0.1
  for the last three points.}
  \label{fig:LU10}
\end{figure}

The solution after 10 iterations is closer to the Neville algorithm solution, 
and the error is a lot lower except for the last points.

\subsection{Question 2d}

Here I use \texttt{time.it} to time how long it takes to execute the functions from (a), (b), and (c).
I run the functions 50 times and get the following average run times:

\lstinputlisting{Timesoutput.txt}

Here are the times in seconds, where (a) and (c) take about 0.004 s to run, and (b) 0.5 s.
LU decomposition and iteration is about a 100 times faster than Neville's algorithm, 
because Neville's algorithm already iterates over the solution to get a more accurate result. 
Doing the LU decomposition and iterating over the solution is faster because you do not have to 
recalculate the LU matrix, and for Neville's algorithm we use bisection for every single point 
we want to interpolate, which adds to the time if we want to interpolate many points with a 
big data set and create a high order polynomial.


